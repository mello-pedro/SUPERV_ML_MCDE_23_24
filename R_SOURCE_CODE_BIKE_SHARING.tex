% Options for packages loaded elsewhere
\PassOptionsToPackage{unicode}{hyperref}
\PassOptionsToPackage{hyphens}{url}
\PassOptionsToPackage{dvipsnames,svgnames,x11names}{xcolor}
%
\documentclass[
  letterpaper,
  DIV=11,
  numbers=noendperiod]{scrartcl}

\usepackage{amsmath,amssymb}
\usepackage{iftex}
\ifPDFTeX
  \usepackage[T1]{fontenc}
  \usepackage[utf8]{inputenc}
  \usepackage{textcomp} % provide euro and other symbols
\else % if luatex or xetex
  \usepackage{unicode-math}
  \defaultfontfeatures{Scale=MatchLowercase}
  \defaultfontfeatures[\rmfamily]{Ligatures=TeX,Scale=1}
\fi
\usepackage{lmodern}
\ifPDFTeX\else  
    % xetex/luatex font selection
\fi
% Use upquote if available, for straight quotes in verbatim environments
\IfFileExists{upquote.sty}{\usepackage{upquote}}{}
\IfFileExists{microtype.sty}{% use microtype if available
  \usepackage[]{microtype}
  \UseMicrotypeSet[protrusion]{basicmath} % disable protrusion for tt fonts
}{}
\makeatletter
\@ifundefined{KOMAClassName}{% if non-KOMA class
  \IfFileExists{parskip.sty}{%
    \usepackage{parskip}
  }{% else
    \setlength{\parindent}{0pt}
    \setlength{\parskip}{6pt plus 2pt minus 1pt}}
}{% if KOMA class
  \KOMAoptions{parskip=half}}
\makeatother
\usepackage{xcolor}
\setlength{\emergencystretch}{3em} % prevent overfull lines
\setcounter{secnumdepth}{-\maxdimen} % remove section numbering
% Make \paragraph and \subparagraph free-standing
\ifx\paragraph\undefined\else
  \let\oldparagraph\paragraph
  \renewcommand{\paragraph}[1]{\oldparagraph{#1}\mbox{}}
\fi
\ifx\subparagraph\undefined\else
  \let\oldsubparagraph\subparagraph
  \renewcommand{\subparagraph}[1]{\oldsubparagraph{#1}\mbox{}}
\fi


\providecommand{\tightlist}{%
  \setlength{\itemsep}{0pt}\setlength{\parskip}{0pt}}\usepackage{longtable,booktabs,array}
\usepackage{calc} % for calculating minipage widths
% Correct order of tables after \paragraph or \subparagraph
\usepackage{etoolbox}
\makeatletter
\patchcmd\longtable{\par}{\if@noskipsec\mbox{}\fi\par}{}{}
\makeatother
% Allow footnotes in longtable head/foot
\IfFileExists{footnotehyper.sty}{\usepackage{footnotehyper}}{\usepackage{footnote}}
\makesavenoteenv{longtable}
\usepackage{graphicx}
\makeatletter
\def\maxwidth{\ifdim\Gin@nat@width>\linewidth\linewidth\else\Gin@nat@width\fi}
\def\maxheight{\ifdim\Gin@nat@height>\textheight\textheight\else\Gin@nat@height\fi}
\makeatother
% Scale images if necessary, so that they will not overflow the page
% margins by default, and it is still possible to overwrite the defaults
% using explicit options in \includegraphics[width, height, ...]{}
\setkeys{Gin}{width=\maxwidth,height=\maxheight,keepaspectratio}
% Set default figure placement to htbp
\makeatletter
\def\fps@figure{htbp}
\makeatother

\usepackage{booktabs}
\usepackage{longtable}
\usepackage{array}
\usepackage{multirow}
\usepackage{wrapfig}
\usepackage{float}
\usepackage{colortbl}
\usepackage{pdflscape}
\usepackage{tabu}
\usepackage{threeparttable}
\usepackage{threeparttablex}
\usepackage[normalem]{ulem}
\usepackage{makecell}
\usepackage{xcolor}
\usepackage[auth-lg]{authblk}
\KOMAoption{captions}{tableheading}
\makeatletter
\@ifpackageloaded{caption}{}{\usepackage{caption}}
\AtBeginDocument{%
\ifdefined\contentsname
  \renewcommand*\contentsname{Table of contents}
\else
  \newcommand\contentsname{Table of contents}
\fi
\ifdefined\listfigurename
  \renewcommand*\listfigurename{List of Figures}
\else
  \newcommand\listfigurename{List of Figures}
\fi
\ifdefined\listtablename
  \renewcommand*\listtablename{List of Tables}
\else
  \newcommand\listtablename{List of Tables}
\fi
\ifdefined\figurename
  \renewcommand*\figurename{Figure}
\else
  \newcommand\figurename{Figure}
\fi
\ifdefined\tablename
  \renewcommand*\tablename{Table}
\else
  \newcommand\tablename{Table}
\fi
}
\@ifpackageloaded{float}{}{\usepackage{float}}
\floatstyle{ruled}
\@ifundefined{c@chapter}{\newfloat{codelisting}{h}{lop}}{\newfloat{codelisting}{h}{lop}[chapter]}
\floatname{codelisting}{Listing}
\newcommand*\listoflistings{\listof{codelisting}{List of Listings}}
\makeatother
\makeatletter
\makeatother
\makeatletter
\@ifpackageloaded{caption}{}{\usepackage{caption}}
\@ifpackageloaded{subcaption}{}{\usepackage{subcaption}}
\makeatother
\ifLuaTeX
  \usepackage{selnolig}  % disable illegal ligatures
\fi
\usepackage{bookmark}

\IfFileExists{xurl.sty}{\usepackage{xurl}}{} % add URL line breaks if available
\urlstyle{same} % disable monospaced font for URLs
\hypersetup{
  pdftitle={PROJETO FINAL - APRENDIZAGEM SUPERVISIONADA},
  pdfauthor={Pedro Henrique Mello Pereira - 230353025; Bernardo Miguel Esperança Sousa - 230353006},
  colorlinks=true,
  linkcolor={blue},
  filecolor={Maroon},
  citecolor={Blue},
  urlcolor={Blue},
  pdfcreator={LaTeX via pandoc}}

\title{PROJETO FINAL - APRENDIZAGEM SUPERVISIONADA}
\usepackage{etoolbox}
\makeatletter
\providecommand{\subtitle}[1]{% add subtitle to \maketitle
  \apptocmd{\@title}{\par {\large #1 \par}}{}{}
}
\makeatother
\subtitle{Bike Sharing Dataset(Regressão Linear Múltipla)}
\author{Pedro Henrique Mello Pereira - 230353025 \and Bernardo Miguel
Esperança Sousa - 230353006}
\date{}

\begin{document}
\maketitle

\renewcommand*\contentsname{Table of contents}
{
\hypersetup{linkcolor=}
\setcounter{tocdepth}{3}
\tableofcontents
}
\newpage{}

\section{Introdução:}\label{introduuxe7uxe3o}

Este trabalho visa desenvolver um modelo de regressão linear múltipla
para prever o número de aluguéis de bicicleta por dia com base em
condições ambientais e sazonais. A análise será realizada utilizando o
dataset ``bikesharing'' e tendo como consideração sua sub-divisão
``day'', não tendo sido feito portanto com base no dataset ``hour''.
Neste estudo, pretendemos explorar como diferentes variáveis, como
estação do ano, ano, mês, feriado, dia da semana e condições
meteorológicas, influenciam o número de aluguéis de bicicleta.

Além disso, é importante destacar que este trabalho será fundamentado na
análise estatística do modelo de regressão linear múltipla. Através
desta abordagem, pretendemos identificar quais variáveis independentes
têm uma influência significativa no número de aluguéis de bicicleta, bem
como avaliar a força e a direção dessas relações. Utilizaremos técnicas
estatísticas para ajustar o modelo aos dados, testar sua adequação e
interpretar os resultados. Ao compreendermos melhor como as variáveis
ambientais e sazonais impactam a demanda por aluguéis de bicicleta,
poderemos fornecer insights valiosos para empresas e organizações
envolvidas no compartilhamento de bicicletas.

\subsection{Definição dos
objetivos:}\label{definiuxe7uxe3o-dos-objetivos}

O principal objetivo deste trabalho é desenvolver um modelo de regressão
linear múltipla que seja capaz de prever o número de aluguéis de
bicicleta por dia com base nas variáveis ambientais e sazonais
fornecidas no conjunto de dados. Pretende-se entender a relação de
fatores como estação do ano, condições meteorológicas, feriados e dia da
semana com a demanda por aluguéis de bicicleta.

Assim sendo, a variável que queremos prever é a ``cnt'' (contagem total
de bicicletas alugadas, incluindo tanto as casuais quanto as
registradas), que neste estudo será chamada de variável dependente e por
vezes também pode assumir as nomenclaturas de variável resposta ou
variável-alvo (target) e que pode ser compreendida como ``a contagem
total de bicicletas alugadas por dia, incluindo utilizadores casuais e
registados'';

\subsection{Apresentação do conjunto de dados e identificação das
variáveis dependente e
independentes:}\label{apresentauxe7uxe3o-do-conjunto-de-dados-e-identificauxe7uxe3o-das-variuxe1veis-dependente-e-independentes}

O dataset utilizado no presente estudo contém a contagem diária de
bicicletas alugadas entre os anos de 2011 e 2012 no sistema de
compartilhamento de bicicletas Capital Bikeshare, juntamente com as
informações meteorológicas e sazonais correspondentes.

A seguir apresentamos o dicionário das variáveis presentes no conjunto
de dados, as quais utilizaremos como variáveis independentes/preditoras
para prever o valor de ``cnt'', a saber:

\begin{itemize}
  \item \textbf{instant}: Índice do registo.
  \item \textbf{dteday}: Data.
  \item \textbf{season}: Estação (1: inverno, 2: primavera, 3: verão, 4: outono).
  \item \textbf{yr}: Ano (0: 2011, 1: 2012).
  \item \textbf{mnth}: Mês (1 a 12).
  \item \textbf{hr}: Hora (0 a 23).
  \item \textbf{holiday}: Dia de tempo é feriado ou não (extraído de http://dchr.dc.gov/page/holiday-schedule).
  \item \textbf{weekday}: Dia da semana.
  \item \textbf{workingday}: Se o dia não é fim de semana nem feriado, é 1, caso contrário é 0.
  \item \textbf{weathersit}:
    \begin{itemize}
      \item 1: Limpo, Poucas nuvens, Parcialmente nublado, Parcialmente nublado.
      \item 2: Nevoeiro + Nublado, Nevoeiro + Nuvens quebradas, Nevoeiro + Poucas nuvens, Nevoeiro.
      \item 3: Neve fraca, Chuva fraca + Trovoada + Nuvens dispersas, Chuva fraca + Nuvens dispersas.
      \item 4: Chuva forte + Granizo + Trovoada + Nevoeiro, Neve + Nevoeiro.
    \end{itemize}
  \item \textbf{temp}: Temperatura normalizada em Celsius. Os valores são derivados via $(t - t\textunderscore{\text{min}})/(t\textunderscore{\text{max}} - t\textunderscore{\text{min}})$, $t\textunderscore{\text{min}}=-8$, $t\textunderscore{\text{max}}=+39$ (apenas em escala horária).
  \item \textbf{atemp}: Sensação térmica normalizada em Celsius. Os valores são derivados via $(t - t\textunderscore{\text{min}})/(t\textunderscore{\text{max}} - t\textunderscore{\text{min}})$, $t\textunderscore{\text{min}}=-16$, $t\textunderscore{\text{max}}=+50$ (apenas em escala horária).
  \item \textbf{hum}: Humidade normalizada. Os valores são divididos por 100 (máximo).
  \item \textbf{windspeed}: Velocidade do vento normalizada. Os valores são divididos por 67 (máximo).
  \item \textbf{casual}: Contagem de utilizadores casuais.
  \item \textbf{registered}: Contagem de utilizadores registados.
  \item \textbf{cnt}: Contagem total de bicicletas alugadas, incluindo utilizadores casuais e registados.
\end{itemize}

\subsubsection{Visão das 5 primeiras entradas do Data set bike
sharing:}\label{visuxe3o-das-5-primeiras-entradas-do-data-set-bike-sharing}

\begingroup\fontsize{2.5}{4.5}\selectfont

\begin{longtable*}{r>{\raggedright\arraybackslash}p{1cm}rrrrrrrrrrrrrr}
\toprule
instant & dteday & season & yr & mnth & holiday & weekday & workingday & weathersit & temp & atemp & hum & windspeed & casual & registered & cnt\\
\midrule
\cellcolor{gray!15}{1} & \cellcolor{gray!15}{2011-01-01} & \cellcolor{gray!15}{1} & \cellcolor{gray!15}{0} & \cellcolor{gray!15}{1} & \cellcolor{gray!15}{0} & \cellcolor{gray!15}{6} & \cellcolor{gray!15}{0} & \cellcolor{gray!15}{2} & \cellcolor{gray!15}{0.344167} & \cellcolor{gray!15}{0.363625} & \cellcolor{gray!15}{0.805833} & \cellcolor{gray!15}{0.1604460} & \cellcolor{gray!15}{331} & \cellcolor{gray!15}{654} & \cellcolor{gray!15}{985}\\
2 & 2011-01-02 & 1 & 0 & 1 & 0 & 0 & 0 & 2 & 0.363478 & 0.353739 & 0.696087 & 0.2485390 & 131 & 670 & 801\\
\cellcolor{gray!15}{3} & \cellcolor{gray!15}{2011-01-03} & \cellcolor{gray!15}{1} & \cellcolor{gray!15}{0} & \cellcolor{gray!15}{1} & \cellcolor{gray!15}{0} & \cellcolor{gray!15}{1} & \cellcolor{gray!15}{1} & \cellcolor{gray!15}{1} & \cellcolor{gray!15}{0.196364} & \cellcolor{gray!15}{0.189405} & \cellcolor{gray!15}{0.437273} & \cellcolor{gray!15}{0.2483090} & \cellcolor{gray!15}{120} & \cellcolor{gray!15}{1229} & \cellcolor{gray!15}{1349}\\
4 & 2011-01-04 & 1 & 0 & 1 & 0 & 2 & 1 & 1 & 0.200000 & 0.212122 & 0.590435 & 0.1602960 & 108 & 1454 & 1562\\
\cellcolor{gray!15}{5} & \cellcolor{gray!15}{2011-01-05} & \cellcolor{gray!15}{1} & \cellcolor{gray!15}{0} & \cellcolor{gray!15}{1} & \cellcolor{gray!15}{0} & \cellcolor{gray!15}{3} & \cellcolor{gray!15}{1} & \cellcolor{gray!15}{1} & \cellcolor{gray!15}{0.226957} & \cellcolor{gray!15}{0.229270} & \cellcolor{gray!15}{0.436957} & \cellcolor{gray!15}{0.1869000} & \cellcolor{gray!15}{82} & \cellcolor{gray!15}{1518} & \cellcolor{gray!15}{1600}\\
\addlinespace
6 & 2011-01-06 & 1 & 0 & 1 & 0 & 4 & 1 & 1 & 0.204348 & 0.233209 & 0.518261 & 0.0895652 & 88 & 1518 & 1606\\
\bottomrule
\end{longtable*}
\endgroup{}

\subsubsection{Metadados do data set objeto do presente
estudo:}\label{metadados-do-data-set-objeto-do-presente-estudo}

Usando a função ``introduce'' do pacote DataExplorer, teremos um
inmportante panorama geral sobre o conjunto de dados bike sharing:

\begin{table}[!h]
\centering\begingroup\fontsize{8}{10}\selectfont

\begin{tabular}{lr}
\toprule
\cellcolor{gray!15}{rows} & \cellcolor{gray!15}{731}\\
columns & 16\\
\cellcolor{gray!15}{discrete\_columns} & \cellcolor{gray!15}{1}\\
continuous\_columns & 15\\
\cellcolor{gray!15}{all\_missing\_columns} & \cellcolor{gray!15}{0}\\
\addlinespace
total\_missing\_values & 0\\
\cellcolor{gray!15}{complete\_rows} & \cellcolor{gray!15}{731}\\
total\_observations & 11696\\
\cellcolor{gray!15}{memory\_usage} & \cellcolor{gray!15}{112208}\\
\bottomrule
\end{tabular}
\endgroup{}
\end{table}

\begin{center}
\includegraphics{R_SOURCE_CODE_BIKE_SHARING_files/figure-pdf/unnamed-chunk-4-1.pdf}
\end{center}

Como podemos perceber, trata-se de um data set de dimensões
relativamente pequenas, a saber 731 linhas (observações) e 16 colunas.

Um dado importante trazido no gráfico de missing values é que não existe
nenhuma entrada com valores ausentes, o que constitui um importante
indicador de qualidade em relação ao preenchimento dos dados.

Na próxima seção nos ocuparemos em limpar os dados, sobretudo para
remover colunas que não fazem sentido e tornar tudo mais claro para a
etapa de análise exploratória de dados e posterior desenvolvimento dos
modelos de regressão linear.

\subsection{Limpeza dos dados:}\label{limpeza-dos-dados}

Em primeiro lugar, removemos as colunas casual e registered, vez que tal
condição fora mencionada como obrigatória no guião do projeto. Desta
feita, excluímos as colunas ora mencionadas pois não serão objeto deste
estudo.

Em seguida, é digno de atenção que cada linha do dataframe em questão
diz respeito a um dia do ano, indo desde 01-01-2011 a 31-12-2012.
Ademais, a variável `instant' servia como indíce.

Logo, com vistas a melhor organizar o data set e diminuir o tempo de
processamento para tarefas de regressão, decidimos por excluir a coluna
`instant' e utilizar a coluna `dteday' como índice.

Assim está a nova configuração dos dados:

\begin{table}[!h]
\centering\begingroup\fontsize{10}{12}\selectfont

\begin{tabular}{lr}
\toprule
\cellcolor{gray!15}{rows} & \cellcolor{gray!15}{731}\\
columns & 13\\
\cellcolor{gray!15}{discrete\_columns} & \cellcolor{gray!15}{1}\\
continuous\_columns & 12\\
\cellcolor{gray!15}{all\_missing\_columns} & \cellcolor{gray!15}{0}\\
\addlinespace
total\_missing\_values & 0\\
\cellcolor{gray!15}{complete\_rows} & \cellcolor{gray!15}{731}\\
total\_observations & 9503\\
\cellcolor{gray!15}{memory\_usage} & \cellcolor{gray!15}{102928}\\
\bottomrule
\end{tabular}
\endgroup{}
\end{table}

\paragraph{Busca por outliers:}\label{busca-por-outliers}

Agora que já excluímos da análise as colunas desnecessárias e checamos a
não existência de valores nulos, é importante atentarmo-nos para a
existência de outliers, que são valores que se afastam
significativamente da maioria dos outros valores num conjunto de dados.
Eles podem distorcer análises estatísticas e modelos, influenciando os
resultados.

Além disso, uma das maiores desvantagens da utilização de modelos de
regressão linear é a sua alta sensibilidade a outliers. Isto posto,
identificar e lidar com outliers é de suma importância para garantir a
precisão do modelo a ser desenvolvido.

\begin{figure}[H]

{\centering \includegraphics{R_SOURCE_CODE_BIKE_SHARING_files/figure-pdf/unnamed-chunk-7-1.pdf}

}

\caption{Distribuições com Outliers}

\end{figure}%

\begin{figure}[H]

{\centering \includegraphics{R_SOURCE_CODE_BIKE_SHARING_files/figure-pdf/unnamed-chunk-7-2.pdf}

}

\caption{Distribuições com Outliers}

\end{figure}%

Conforme depreende-se da análise dos boxplots, as variáveis `hum' e
`windspeed' apresentaram outliers em sua composição. Para lidar com este
problema, haja vista serem poucos os valores classificados como
outliers, excluiremos tais valores discrepantes do data set
df\_bike\_new.

\paragraph{Exclusão dos outliers por meio da definição dos limites
superiores e inferiores pelo IQR
SCORE:}\label{exclusuxe3o-dos-outliers-por-meio-da-definiuxe7uxe3o-dos-limites-superiores-e-inferiores-pelo-iqr-score}

A exclusão dos outliers será feita por meio do IQR Score. O IQR
(Intervalo Interquartil) é uma medida de dispersão que indica a
amplitude dos dados em torno da mediana. O IQR score é uma medida
estatística usada para identificar outliers, calculada como a diferença
entre o terceiro quartil (Q3) e o primeiro quartil (Q1). Outliers
geralmente são definidos como valores que caem abaixo de Q1 - 1,5 * IQR
(Limite inferior) ou acima de Q3 + 1,5 * IQR (Limite superior).

Após realizarmos a exclusão dos outliers, apenas 18 observações foram
apagadas do conjunto de dados. Apesar de ser interessante contar com o
máximo de dados possíveis, é ainda mais importante para o nosso modelo a
não existência de outliers.

Utlizamos também o parâmetro distinct do R para eliminar duplicatas do
nosso conjunto de dados. Entretanto, nenhuma duplicata foi identificada.

Após a finalização da limpeza dos dados, vamos salvá-los em um novo
ficheiro o qual chamaremos de ``df\_clean'' e que será utilizado de
agora em diante na análise exploratória e modelagem.

\subsection{Análise Exploratória dos
Dados}\label{anuxe1lise-exploratuxf3ria-dos-dados}

Ao implementar as etapas de limpeza anteriores, criamos um conjunto de
dados mais refinado e confiável, estabelecendo a base para nossa
subsequente análise exploratória de dados (AED), bem como para o
desenvolvimento de modelos preditivos. O tratamento cuidadoso dos
problemas de qualidade dos dados é crucial para garantir a precisão e
confiabilidade dos insights que serão derivados do conjunto de dados.

Prosseguindo, nosso foco irá mudar para a Análise Exploratória de Dados
(AED) para obter insights mais profundos sobre os dados de
compartilhamento de bicicletas e identificar padrões acionáveis. Esta
fase analítica tem como objetivo descobrir tendências significativas,
correlações e padrões dentro do conjunto de dados.

A primeira medida será tranformar em fator as variáveis categóricas,
utilizando para tanto a função factor do R.

\subsubsection{Explorando a distribuição das variáveis independentes
(numéricas):}\label{explorando-a-distribuiuxe7uxe3o-das-variuxe1veis-independentes-numuxe9ricas}

\begin{center}
\includegraphics{R_SOURCE_CODE_BIKE_SHARING_files/figure-pdf/unnamed-chunk-11-1.pdf}
\end{center}

\begin{center}
\includegraphics{R_SOURCE_CODE_BIKE_SHARING_files/figure-pdf/unnamed-chunk-11-2.pdf}
\end{center}

\begin{center}
\includegraphics{R_SOURCE_CODE_BIKE_SHARING_files/figure-pdf/unnamed-chunk-11-3.pdf}
\end{center}

\begin{center}
\includegraphics{R_SOURCE_CODE_BIKE_SHARING_files/figure-pdf/unnamed-chunk-11-4.pdf}
\end{center}

\begin{itemize}
\item
  Quanto à variável que diz respeito à umidade relativa, a distribuição
  aproxima-se da normalidade.
\item
  Já em relação à distribuição das variáveis ``sensação térmica'' e
  ``temperatura'', podemos dizer que a análise dos gráficos de
  distribuição nos leva a concluir que pode existir uma distribuição
  bimodal em amos os casos.
\item
  A variável relativa à velocidade do vento, por sua vez, traz uma
  assimetria positiva em sua distribuição.
\end{itemize}

\subsubsection{Explorando a relação entre o número de bikes alugadas e a
estação do
ano:}\label{explorando-a-relauxe7uxe3o-entre-o-nuxfamero-de-bikes-alugadas-e-a-estauxe7uxe3o-do-ano}

\begin{center}
\includegraphics{R_SOURCE_CODE_BIKE_SHARING_files/figure-pdf/unnamed-chunk-12-1.pdf}
\end{center}

Aqui podemos perceber que há uma clara tendência pela maior procura de
aluguel de bicicletas na primavera e verão, tendo como contraponto o
inverno, estação em que os alugueis reduzem-se a um terço do observado
no verão.

\subsubsection{Explorando a relação entre o número de bikes alugadas e o
ano:}\label{explorando-a-relauxe7uxe3o-entre-o-nuxfamero-de-bikes-alugadas-e-o-ano}

\begin{center}
\includegraphics{R_SOURCE_CODE_BIKE_SHARING_files/figure-pdf/unnamed-chunk-13-1.pdf}
\end{center}

Conforme depreende-se da análise do gráfico, houve um aumento
substancial no número de bicicletas no ano de 2012, se comparado ao ano
anterior.

Isto pode significar uma tendência de crescimento para este mercado.
Para confirmar tal asuncão seria necessário ter os dados dos anos
posteriores e proceder à análise de séries temporais para saber se a
tendência se confirma.

\subsubsection{E em relação aos
meses?}\label{e-em-relauxe7uxe3o-aos-meses}

\begin{center}
\includegraphics{R_SOURCE_CODE_BIKE_SHARING_files/figure-pdf/unnamed-chunk-14-1.pdf}
\end{center}

Tendo em consideração a distribuição das estações climáticas para os
países do hemisfério norte, podemos depreender da análise do gráfico
acima uma confirmação da constatação relativa ao primeiro ponto desta
análise exploratória, onde observamos uma tendência muito maior ao
aluguel de bicicletas durante a primareva e o verão.

Aqui a tendência se confirma, demonstrando que nos meses mais quentes há
um aumento sensível no número de bicicletas alugadas.

\subsubsection{Explorando a relação entre o número de bikes alugadas e o
dia da
semana:}\label{explorando-a-relauxe7uxe3o-entre-o-nuxfamero-de-bikes-alugadas-e-o-dia-da-semana}

\begin{center}
\includegraphics{R_SOURCE_CODE_BIKE_SHARING_files/figure-pdf/unnamed-chunk-15-1.pdf}
\end{center}

\newpage{}

Em relação ao dia da semana, a análise do gráfico nos mostra que há
quase um equilíbrio na distribuição ao longo dos dias, com uma tendência
um pouco menor de aluguéis aos domingos e segundas-feiras, bem como uma
leve tendência a um maior número de alugueis aos sábados.

Desta feita, caso a empresa queira aumentar o número de alugueis no
domingo/segunda-feira com vistas a igualar os demais dias da semana,
pode ser interessante endereçar campanhas de marketing com descontos ou
programas de fidelidade destinados a estes dias em específico.

\subsubsection{E como comportam-se os números de alugueis de bicicletas
em relação à condição
climática?}\label{e-como-comportam-se-os-nuxfameros-de-alugueis-de-bicicletas-em-relauxe7uxe3o-uxe0-condiuxe7uxe3o-climuxe1tica}

\begin{center}
\includegraphics{R_SOURCE_CODE_BIKE_SHARING_files/figure-pdf/unnamed-chunk-16-1.pdf}
\end{center}

Conforme era de se esperar, a maioria dos alugueis acontece quando as
condições climáticas são: Limpo, Poucas nuvens ou Parcialmente nublado.

Em seguida sob as condições ``Nevoeiro + Nublado, Nevoeiro + Nuvens
quebradas, Nevoeiro + Poucas nuvens, Nevoeiro'', os alugueis caem para
menos da metade em relação à condição número 1.

É importante termos atenção à variável independente `weathersit' pois
pode ser que ela contribua sobremaneira para explicar as variações em
`cnt'.

\section{Desenvolvimento do Modelo De Regressão Linear
Múltipla}\label{desenvolvimento-do-modelo-de-regressuxe3o-linear-muxfaltipla}

A regressão linear múltipla é uma técnica estatística que busca modelar
a relação entre uma variável dependente, que é aquela que queremos
prever, e duas ou mais variáveis independentes. Ela é uma extensão da
regressão linear simples, que envolve apenas uma variável independente.
Na regressão linear múltipla, o objetivo é estimar os coeficientes das
variáveis independentes para prever ou explicar a variabilidade na
variável dependente.

A equação da regressão linear múltipla é representada a seguir:

\(y = \beta_0 + \beta_1 x_1 + \beta_2 x_2 + \ldots + \beta_p x_p + \varepsilon\)

Após a fase inicial de carregamento e limpeza dos dados, assim como uma
análise exploratória para compreender melhor as características do
conjunto de dados, avançaremos para a etapa de desenvolvimento de
modelos de regressão linear múltipla. Esta etapa é crucial para o
projeto, pois visa prever o número de bicicletas alugadas (\texttt{cnt})
com base em diversas variáveis explicativas disponíveis. Neste contexto,
será adotada a métrica de avaliação do coeficiente de determinação
ajustado (R² ajustado), dada sua relevância para a precisão de modelos
de regressão linear múltipla.

O coeficiente de determinação ajustado (R² ajustado) é uma medida
estatística que avalia o quão bem o modelo de regressão linear múltipla
se ajusta aos dados, levando em consideração o número de variáveis
independentes incluídas no modelo. Ele é uma versão ajustada do
coeficiente de determinação (R²), que quantifica a proporção da
variabilidade na variável dependente que é explicada pelo modelo.

A fórmula do coeficiente de determinação ajustado é representada a
seguir:

\(R^2_{ajustado} = 1 - \frac{RSS / (n - p - 1)}{TSS / (n - 1)}\)

onde:

\begin{itemize}
\tightlist
\item
  (RSS) é a soma dos quadrados dos resíduos (erro quadrático médio
  residual).
\item
  (TSS) é a soma total dos quadrados.
\item
  (n) é o número total de observações.
\item
  (p) é o número de variáveis independentes no modelo.
\end{itemize}

Durante o processo de modelagem, serão exploradas algumas combinações de
variáveis, incluindo algumas que foram identificadas durante a análise
exploratória como potencialmente influentes no número de alugueis de
bicicletas. Esta abordagem permitirá testar a influência relativa de
cada variável sobre a variável de interesse (\texttt{cnt}). Além disso,
serão consideradas iterações do modelo, incluindo ou excluindo
variáveis, para determinar qual combinação oferece o melhor desempenho
com base no R² ajustado. Este procedimento visa encontrar um equilíbrio
entre a simplicidade do modelo e sua capacidade de explicar a variação
nos dados observados.

\subsection{Pressupostos para a validação de um modelo de Regressão
Linear
Múltipla:}\label{pressupostos-para-a-validauxe7uxe3o-de-um-modelo-de-regressuxe3o-linear-muxfaltipla}

Para que seja considerado como estatisticamente significativo e
apresente resultados com qualidade e confiablidade, um modelo de
regressão linear deve reunir alguns pressupostos básicos. Não
abordaremos todos em seus mínimos detalhes neste estudo, portanto
partiremos do pressuposto que existe relação linear entre a variável
dependente ``cnt'' e ao menos uma das variáveis independentes.

\begin{enumerate}
\def\labelenumi{\arabic{enumi})}
\tightlist
\item
  Não existência de Multicolinearidade entre as variaveis preditoras.
\end{enumerate}

Antes de avançar, é importante plotar a matriz de correlação para saber
se não há multicolinearidade entre as nossas variáveis preditoras. Desta
feita, no primeiro momento analisaremos apenas a matriz de correlação de
Pearson para avaliar se há correlação linear entre as nossas variáveis
numéricas e o quão forte ela é.

\begin{center}
\includegraphics{R_SOURCE_CODE_BIKE_SHARING_files/figure-pdf/unnamed-chunk-17-1.pdf}
\end{center}

Salta aos olhos a forte correlação linear entre as variáveis temp e
atemp. Ora, sendo temp a temperatura real medida naquela data e atemp a
sensação térmica, fica fácil perceber a razão desta correlação linear
positiva tão forte.

Assim sendo, com vistas a evitar a violação do pressuposto da não
existência de multicolinearidade entre as variáveis independentes no
modelo de regressão linear, excluíremos da nossa análise a variável
``atemp'', mantendo apenas ``temp'' no modelo.

Quanto às variáveis categóricas, que constituem uma importante parte das
variáveis independentes do nosso modelo, analisaremos se existe
multicolinearidade entre elas mais adiante, após a construção do modelo,
por meio do VIF (Variance Inflation Factor). Por ora, o que sabemos é
que a variável preditora ``atemp'' de partida já não estará presente nos
dados.

\begin{enumerate}
\def\labelenumi{\arabic{enumi})}
\setcounter{enumi}{1}
\item
  Os resíduos devem ter média = 0 e distribuição normal.
\item
  Homocedasticidade das variâncias dos erros. Partiremos do princípio
  que este pressuposto está satisfeito para os dados em questão.
\end{enumerate}

\subsubsection{\texorpdfstring{\textbf{Desenvolvendo e avaliando o
Modelo 1 de Regressão Linear
Múltipla:}}{Desenvolvendo e avaliando o Modelo 1 de Regressão Linear Múltipla:}}\label{desenvolvendo-e-avaliando-o-modelo-1-de-regressuxe3o-linear-muxfaltipla}

Desenvolveremos a seguir um modelo de regressão Linear Múltipla saturado
com todas as variáveis independentes que temos à disposição, à exceção
de ``atemp'' que foi eliminada na etapa anterior.

O objetivo aqui é avaliar a acurácia do modelo, sobretudo no que diz
respeito à sifnificância e grau de contribuição de cada variável
independente para a assertividade das previsões.

Ressaltamos que todas as análises realizadas a partir deste momento
terão em consideração um nível de significância \(\alpha = 0.05\).

\begin{verbatim}

Call:
lm(formula = cnt ~ ., data = train_data)

Residuals:
    Min      1Q  Median      3Q     Max 
-3320.2  -339.6    66.3   439.7  2522.2 

Coefficients: (1 not defined because of singularities)
             Estimate Std. Error t value Pr(>|t|)    
(Intercept) 83356.785  52998.883   1.573 0.116343    
dteday         -5.451      3.537  -1.541 0.123799    
season2       889.497    208.418   4.268 2.33e-05 ***
season3       800.525    243.107   3.293 0.001056 ** 
season4      1545.817    204.540   7.558 1.74e-13 ***
yr1          3957.887   1294.695   3.057 0.002345 ** 
mnth2         320.500    184.566   1.737 0.083038 .  
mnth3         797.331    278.345   2.865 0.004336 ** 
mnth4         967.271    419.089   2.308 0.021369 *  
mnth5        1358.431    514.285   2.641 0.008493 ** 
mnth6        1276.007    613.166   2.081 0.037897 *  
mnth7         982.290    719.175   1.366 0.172545    
mnth8        1619.025    813.581   1.990 0.047088 *  
mnth9        2295.850    902.711   2.543 0.011256 *  
mnth10       2072.195   1002.698   2.067 0.039240 *  
mnth11       1646.744   1107.966   1.486 0.137782    
mnth12       1773.898   1200.361   1.478 0.140036    
holiday1     -683.278    196.741  -3.473 0.000555 ***
weekday1      254.329    122.514   2.076 0.038369 *  
weekday2      380.692    116.303   3.273 0.001130 ** 
weekday3      472.973    115.479   4.096 4.84e-05 ***
weekday4      376.212    117.433   3.204 0.001436 ** 
weekday5      515.156    117.820   4.372 1.47e-05 ***
weekday6      468.829    117.845   3.978 7.87e-05 ***
workingday1        NA         NA      NA       NA    
weathersit2  -499.501     85.406  -5.849 8.54e-09 ***
weathersit3 -2040.012    232.645  -8.769  < 2e-16 ***
temp         4466.890    462.041   9.668  < 2e-16 ***
hum         -1673.201    337.446  -4.958 9.50e-07 ***
windspeed   -2876.164    473.284  -6.077 2.30e-09 ***
---
Signif. codes:  0 '***' 0.001 '**' 0.01 '*' 0.05 '.' 0.1 ' ' 1

Residual standard error: 748.8 on 547 degrees of freedom
Multiple R-squared:  0.8566,    Adjusted R-squared:  0.8493 
F-statistic: 116.7 on 28 and 547 DF,  p-value: < 2.2e-16
\end{verbatim}

\subsubsection{Avaliação geral do desempenho do Modelo
1:}\label{avaliauxe7uxe3o-geral-do-desempenho-do-modelo-1}

Em primeiro medida, analisemos as hipóteses associadas ao F-statistic,
que nos traz a informação a respeito do quão significativo é o modelo,
informando-nos ainda a respeito da existência de ao menos uma variável
independente que seja suficientemente significativa para explicar a
variável-alvo cnt:

\begin{itemize}
\item
  H0: Não há relação significativa entre as variáveis independentes e a
  variável dependente no modelo.
\item
  H1: Pelo menos uma das variáveis independentes tem uma relação
  significativa com a variável dependente no modelo.
\end{itemize}

Logo, haja vista que o valor de F-statistic encontra-se distante de 1
(F-Statistic = 116.7), tendo ainda o p-value associado a este teste um
valor \textless{} \(\alpha\) (p-value: \textless{} 2.2e-16), possuímos
evidências estatísticas suficientes para rejeitar a hipótese nula do
Teste F e afirmar que \textbf{O modelo é globalmente significativo e
pelo menos uma das variáveis independentes tem uma relação significativa
com a variável dependente no modelo.}

\paragraph{Acurácia do modelo e Regressão Linear
Múltipla:}\label{acuruxe1cia-do-modelo-e-regressuxe3o-linear-muxfaltipla}

Vejamos os números relacionados às métricas de avaliação da acurácia e
precisão do modelo preditivo:

\begin{verbatim}
[1] "RMSE (Training): 729.70166960945"
\end{verbatim}

\begin{verbatim}
[1] "MAE (Training): 536.39163608996"
\end{verbatim}

\begin{verbatim}
[1] "R-squared (Training): 0.856648892418335"
\end{verbatim}

\begin{verbatim}
[1] "R-squared (Adjusted, Training): 0.849035005751909"
\end{verbatim}

Na presente análise nos concentraremos apenas nas medidas do Coeficiente
de Determinação Ajustado, haja vista que tal métrica possui um padrão de
análise que varia de 0 a 1, sendo que quanto mais próximo de 0, pior é a
acurácia do modelo e quanto mais próximo de 1 melhor será o seu
desempenho preditivo.

Além disso, o motivo pelo qual consideraremos o Coeficiente de
Determinação Ajustado ao invés do Coeficiente de Determinação
``simples'' é que esta métrica é a mais recomendável para regressões
lineares múltiplas, sobretudo aquelas que possuem muitas variáveis
independentes, como é o caso nesta análise. O coeficiente de Determinção
Ajustado penaliza mais os modelos maiores e que possuem mais variáveis.

Isto posto, \textbf{é possível afirmar que o nosso modelo performou bem
ao realizar predições para os dados de treino, ao apresentar um
R-squared (Adjusted) = 0.849035005751909.}

A seguir, apresentaremos as conclusões relativas à performance do modelo
quando aplicado à predição dos dados de teste:

\begin{verbatim}
[1] "RMSE: 855.10298986998"
\end{verbatim}

\begin{verbatim}
[1] "MAE: 608.153511116337"
\end{verbatim}

\begin{verbatim}
[1] "R-squared: 0.808121594695134"
\end{verbatim}

\begin{verbatim}
[1] "R-squared (Adjusted): 0.757991200516386"
\end{verbatim}

Aqui há um alerta: Ao aplicarmos o modelo aos dados ``novos'' do
conjunto de Teste, o valor do Coeficiente de Determinação Ajustado cai
consideravelmente, vez que neste caso R-squared (Adjusted) =
0.757991200516386.

Isto não quer dizer que seja um modelo ruim, haja vista que 0.75 ainda é
um valor relativamente alto, mas indica que pode haver espaço para
melhorias, sobretudo ao retirar do modelo aquelas variáveis que não
contribuem para explicar a variável-alvo, o que será assunto do próximo
tópico.

\begin{center}
\includegraphics{R_SOURCE_CODE_BIKE_SHARING_files/figure-pdf/unnamed-chunk-21-1.pdf}
\end{center}

\paragraph{Significância das variáveis preditoras para o
modelo:}\label{significuxe2ncia-das-variuxe1veis-preditoras-para-o-modelo}

Ao analisar os resultados do treino do nosso modelo, sob a ótica da
contribuição das variáveis independentes na explicação da variável-alvo,
devemos considerar o teste T associado a elas e o p-value associado,
sendo estas as hipóteses:

\(H_0: \beta_1 = 0\)

\(H_1: \beta_1 \neq 0\)

tendo em consideração o nível de significância definido neste trabalho
(\(\alpha\) = 0.05), consideraremos como significativas para o modelo
(ou seja, rejeita-se a hipótese nula apenas para) as seguintes variáveis
independentes:

\begin{longtable}[]{@{}
  >{\raggedright\arraybackslash}p{(\columnwidth - 10\tabcolsep) * \real{0.1711}}
  >{\raggedright\arraybackslash}p{(\columnwidth - 10\tabcolsep) * \real{0.1447}}
  >{\raggedright\arraybackslash}p{(\columnwidth - 10\tabcolsep) * \real{0.1579}}
  >{\raggedright\arraybackslash}p{(\columnwidth - 10\tabcolsep) * \real{0.1184}}
  >{\raggedright\arraybackslash}p{(\columnwidth - 10\tabcolsep) * \real{0.1316}}
  >{\raggedright\arraybackslash}p{(\columnwidth - 10\tabcolsep) * \real{0.2763}}@{}}
\toprule\noalign{}
\begin{minipage}[b]{\linewidth}\raggedright
Variável
\end{minipage} & \begin{minipage}[b]{\linewidth}\raggedright
Estimate
\end{minipage} & \begin{minipage}[b]{\linewidth}\raggedright
Std. Error
\end{minipage} & \begin{minipage}[b]{\linewidth}\raggedright
t value
\end{minipage} & \begin{minipage}[b]{\linewidth}\raggedright
Pr(\textgreater)
\end{minipage} & \begin{minipage}[b]{\linewidth}\raggedright
Significância
\end{minipage} \\
\midrule\noalign{}
\endhead
\bottomrule\noalign{}
\endlastfoot
weathersit3 & -2040.012 & 232.645 & -8.769 & \textless{} 2e-16 & Muito
significativo \\
temp & 4466.890 & 462.041 & 9.668 & \textless{} 2e-16 & Muito
significativo \\
season4 & 1545.817 & 204.540 & 7.558 & 1.74e-13 & Muito significativo \\
yr1 & 3957.887 & 1294.695 & 3.057 & 0.002345 & Muito significativo \\
windspeed & -2876.164 & 473.284 & -6.077 & 2.30e-09 & Muito
significativo \\
weathersit2 & -499.501 & 85.406 & -5.849 & 8.54e-09 & Muito
significativo \\
holiday1 & -683.278 & 196.741 & -3.473 & 0.000555 & Muito
significativo \\
weekday5 & 515.156 & 117.820 & 4.372 & 1.47e-05 & Muito significativo \\
weekday1 & 254.329 & 122.514 & 2.076 & 0.038369 & significativo \\
season2 & 889.497 & 208.418 & 4.268 & 2.33e-05 & Muito significativo \\
season3 & 800.525 & 243.107 & 3.293 & 0.001056 & Muito significativo \\
weekday2 & 380.692 & 116.303 & 3.273 & 0.001130 & Muito significativo \\
weekday3 & 472.973 & 115.479 & 4.096 & 4.84e-05 & Muito significativo \\
weekday4 & 376.212 & 117.433 & 3.204 & 0.001436 & Muito significativo \\
weekday6 & 468.829 & 117.845 & 3.978 & 7.87e-05 & Muito significativo \\
mnth3 & 797.331 & 278.345 & 2.865 & 0.004336 & Muito significativo \\
mnth4 & 967.271 & 419.089 & 2.308 & 0.021369 & Significativo \\
mnth5 & 1358.431 & 514.285 & 2.641 & 0.008493 & Muito significativo \\
mnth6 & 1276.007 & 613.166 & 2.081 & 0.037897 & Significativo \\
mnth8 & 1619.025 & 813.581 & 1.990 & 0.047088 & Significativo \\
mnth9 & 2295.850 & 902.711 & 2.543 & 0.011256 & Muito significativo \\
mnth10 & 2072.195 & 1002.698 & 2.067 & 0.039240 & Significativo \\
hum & -1673.201 & 337.446 & -4.958 & 9.50e-07 & Muito significativo \\
\end{longtable}

Aquelas que consideramos como pouco ou não significativas de acordo com
o resultado do teste T associado são (incluindo-se o intercepto):

\begin{longtable}[]{@{}
  >{\raggedright\arraybackslash}p{(\columnwidth - 10\tabcolsep) * \real{0.1781}}
  >{\raggedright\arraybackslash}p{(\columnwidth - 10\tabcolsep) * \real{0.1370}}
  >{\raggedright\arraybackslash}p{(\columnwidth - 10\tabcolsep) * \real{0.1644}}
  >{\raggedright\arraybackslash}p{(\columnwidth - 10\tabcolsep) * \real{0.1233}}
  >{\raggedright\arraybackslash}p{(\columnwidth - 10\tabcolsep) * \real{0.1370}}
  >{\raggedright\arraybackslash}p{(\columnwidth - 10\tabcolsep) * \real{0.2603}}@{}}
\toprule\noalign{}
\begin{minipage}[b]{\linewidth}\raggedright
Variável
\end{minipage} & \begin{minipage}[b]{\linewidth}\raggedright
Estimate
\end{minipage} & \begin{minipage}[b]{\linewidth}\raggedright
Std. Error
\end{minipage} & \begin{minipage}[b]{\linewidth}\raggedright
t value
\end{minipage} & \begin{minipage}[b]{\linewidth}\raggedright
Pr(\textgreater)
\end{minipage} & \begin{minipage}[b]{\linewidth}\raggedright
Significância
\end{minipage} \\
\midrule\noalign{}
\endhead
\bottomrule\noalign{}
\endlastfoot
(Intercept) & 83356.785 & 52998.883 & 1.573 & 0.116343 & Não
significativo \\
dteday & -5.451 & 3.537 & -1.541 & 0.123799 & Não significativo \\
workingday1 & NA & NA & NA & NA & NA \\
mnth2 & 320.500 & 184.566 & 1.737 & 0.083038 & Não significativo \\
mnth7 & 982.290 & 719.175 & 1.366 & 0.172545 & Não significativo \\
mnth11 & 1646.744 & 1107.966 & 1.486 & 0.137782 & Não significativo \\
mnth12 & 1773.898 & 1200.361 & 1.478 & 0.140036 & Não significativo \\
\end{longtable}

\newpage{}

\paragraph{E quanto ao pressuposto da distribuição dos
resíduos?}\label{e-quanto-ao-pressuposto-da-distribuiuxe7uxe3o-dos-resuxedduos}

\begin{figure}[H]

{\centering \includegraphics{R_SOURCE_CODE_BIKE_SHARING_files/figure-pdf/unnamed-chunk-22-1.pdf}

}

\caption{Distribuições dos Resíduos}

\end{figure}%

Podemos observar que mesmo com um certo ``peso'' nas caudas, sobretudo
do lado negativo, há uma concentração muito considerável de observações
dos resíduos sobre a curva, o que nos indica que a distribuição destes
resíduos está muito próxima da normal e a média está próxima de 0, vindo
a confirmar portanto a satisfação do pressuposto.

Importante ressaltar que pode ser que os próximos modelos, com menos
variáveis, sejam ainda mais aderentes ao pressuposto e venham a corrigir
um pouco este ``peso'' a maior nas caudas da distribuição dos resíduos.

\subsubsection{Problemas identificados no
modelo}\label{problemas-identificados-no-modelo}

Apesar de apresentar indicadores bons, o primeiro modelo possui alguns
problemas. Um deles é o excesso de variáveis, que costuma ser
computacionalmente custoso a modelos de regressào linear, sobretudo pela
quantidade de variáveis categóricas existentes e que acabam por gerar um
alto número de variáveis dummy.

Assim sendo, pode ser benéfico ao modelo - em questões computacionais e
tendo-se em consideração os possíveis custos para uma organização
relacionados ao deploy deste, a redução da dimensionalidade do conjunto
de variáveis preditoras.

Entretanto, um outro problema ainda mais grave foi diagnisticado: Ao
tentar utilizar o VIF para calcular a existência de multicolinearidade
entre as variáveis independentes do modelo, o R traz o seguinte erro
como output:

\begin{center}
\includegraphics{images/Captura de Tela 2024-03-11 às 12.23.56.png}
\end{center}

Este erro indica que pode existir multicolinearidade entre as variáveis
preditoras do modelo. Ora, sendo a não existência de multicolinearidade
entre as variáveis preditoras um dos principais pressupostos para a
assunção de que um modelo de regressão linear é bom e confiável,
resta-nos a opção de investigar se está a ocorrer multicolinearidade e
excluir as variáveis problemáticas.

\subsubsection{\texorpdfstring{\textbf{Modelo 2 - Novo modelo com menos
variáveis}}{Modelo 2 - Novo modelo com menos variáveis}}\label{modelo-2---novo-modelo-com-menos-variuxe1veis}

Neste momento nos dedicaremos à melhora do modelo anterior, retirando
inicialmente as variáveis ``dteday'' e ``workingday'', baseados em sua
não contribuição e significância e analisaremos o VIF para este caso.

\begin{verbatim}

Call:
lm(formula = cnt ~ . - workingday - dteday, data = train_data)

Residuals:
    Min      1Q  Median      3Q     Max 
-3382.3  -355.7    61.1   444.5  2568.6 

Coefficients:
            Estimate Std. Error t value Pr(>|t|)    
(Intercept)  1665.61     267.73   6.221 9.80e-10 ***
season2       885.19     208.66   4.242 2.60e-05 ***
season3       796.77     243.40   3.273 0.001129 ** 
season4      1555.49     204.70   7.599 1.30e-13 ***
yr1          1964.74      64.74  30.348  < 2e-16 ***
mnth2         164.25     154.43   1.064 0.287985    
mnth3         479.48     187.19   2.561 0.010690 *  
mnth4         487.98     281.33   1.735 0.083382 .  
mnth5         714.68     300.48   2.378 0.017728 *  
mnth6         468.57     319.10   1.468 0.142566    
mnth7          14.21     350.81   0.041 0.967700    
mnth8         477.29     336.96   1.416 0.157208    
mnth9         983.05     299.55   3.282 0.001097 ** 
mnth10        585.06     273.42   2.140 0.032815 *  
mnth11        -14.11     258.37  -0.055 0.956476    
mnth12        -49.11     205.42  -0.239 0.811154    
holiday1     -676.65     196.94  -3.436 0.000636 ***
weekday1      252.80     122.66   2.061 0.039786 *  
weekday2      376.90     116.42   3.237 0.001280 ** 
weekday3      472.34     115.62   4.085 5.06e-05 ***
weekday4      375.12     117.58   3.190 0.001502 ** 
weekday5      514.62     117.97   4.362 1.54e-05 ***
weekday6      467.94     117.99   3.966 8.28e-05 ***
weathersit2  -489.14      85.25  -5.738 1.59e-08 ***
weathersit3 -2041.73     232.93  -8.765  < 2e-16 ***
temp         4447.61     462.45   9.617  < 2e-16 ***
hum         -1700.92     337.39  -5.041 6.29e-07 ***
windspeed   -2897.36     473.68  -6.117 1.82e-09 ***
---
Signif. codes:  0 '***' 0.001 '**' 0.01 '*' 0.05 '.' 0.1 ' ' 1

Residual standard error: 749.7 on 548 degrees of freedom
Multiple R-squared:  0.856, Adjusted R-squared:  0.8489 
F-statistic: 120.7 on 27 and 548 DF,  p-value: < 2.2e-16
\end{verbatim}

Como podemos perceber, o modelo 2 aparenta ter bons indicadores, tendo
um Adjusted R-squared = 0.8489, ou seja, pouquíssimo inferior ao seu
antecessor e com um valor de F-statistic = 120.7, corroborado pelo
p-value \textless{} \(\alpha\).

Aqui o intercepto já passa a ser significativo para o modelo, ao
contrário do que ocorrera anteriormente e a variável
``mnth''(representada pelas variáveis dummy) parece seu pouco
significtiva para explicar a variável resposta ``cnt''.

Entretanto, vejamos o VIF associado ao novo modelo:

\begin{verbatim}
                 GVIF Df GVIF^(1/(2*Df))
season     211.907282  3        2.441693
yr           1.073706  1        1.036198
mnth       522.363646 11        1.329059
holiday      1.138400  1        1.066958
weekday      1.219839  6        1.016698
weathersit   2.001208  2        1.189387
temp         7.458795  1        2.731080
hum          2.209591  1        1.486469
windspeed    1.208740  1        1.099427
\end{verbatim}

Tendo em consideração que um valor de VIF \textgreater{} 10 denota a
existência de multicolinearidade, é certo dizer que as variáveis season
(VIF = 211.907282) e mnth(VIF = 522.363646) possuem multicolinearidade e
fazem com que o modelo viole o pressuposto relacionado à não existência
desta condição.

Chama atenção ainda que a variável ``temp'' apresente um VIF muito
próximo do limite aceitável. Entretanto, na presente análise não a
descartaremos.

Desta feita, desenvolveremos um novo modelo, desta vez sem a variável
``mnth'', haja vista ter apresentado níveis de significância para o
modelo mais modestos que a variável ``season'', bem como o maior VIF se
comparada às demais.

\subsubsection{\texorpdfstring{\textbf{Modelo 3: Excluindo ``mnth'' do
conjunto de variáveis
independentes:}}{Modelo 3: Excluindo ``mnth'' do conjunto de variáveis independentes:}}\label{modelo-3-excluindo-mnth-do-conjunto-de-variuxe1veis-independentes}

\begin{verbatim}

Call:
lm(formula = cnt ~ . - workingday - dteday - mnth, data = train_data)

Residuals:
     Min       1Q   Median       3Q      Max 
-3122.09  -375.11    62.75   505.89  2300.57 

Coefficients:
            Estimate Std. Error t value Pr(>|t|)    
(Intercept)  1389.57     261.71   5.310 1.59e-07 ***
season2      1160.66     126.79   9.154  < 2e-16 ***
season3       857.26     166.30   5.155 3.53e-07 ***
season4      1581.21     105.74  14.953  < 2e-16 ***
yr1          1978.79      67.16  29.466  < 2e-16 ***
holiday1     -729.85     203.73  -3.582 0.000370 ***
weekday1      236.55     128.54   1.840 0.066250 .  
weekday2      375.94     121.89   3.084 0.002141 ** 
weekday3      477.36     120.92   3.948 8.89e-05 ***
weekday4      371.77     122.85   3.026 0.002590 ** 
weekday5      538.35     123.84   4.347 1.64e-05 ***
weekday6      490.39     123.76   3.963 8.38e-05 ***
weathersit2  -506.90      88.64  -5.719 1.75e-08 ***
weathersit3 -2092.16     241.54  -8.662  < 2e-16 ***
temp         4978.92     339.96  14.646  < 2e-16 ***
hum         -1304.42     333.20  -3.915 0.000102 ***
windspeed   -2771.34     492.62  -5.626 2.92e-08 ***
---
Signif. codes:  0 '***' 0.001 '**' 0.01 '*' 0.05 '.' 0.1 ' ' 1

Residual standard error: 788.6 on 559 degrees of freedom
Multiple R-squared:  0.8375,    Adjusted R-squared:  0.8328 
F-statistic: 180.1 on 16 and 559 DF,  p-value: < 2.2e-16
\end{verbatim}

\begin{verbatim}
               GVIF Df GVIF^(1/(2*Df))
season     3.738169  3        1.245785
yr         1.044119  1        1.021821
holiday    1.101030  1        1.049300
weekday    1.162640  6        1.012637
weathersit 1.909766  2        1.175561
temp       3.642921  1        1.908644
hum        1.947716  1        1.395606
windspeed  1.181545  1        1.086989
\end{verbatim}

Ao avaliarmos a performance do modelo, agora sem a presença das
variáveis independentes ``workingday'', ``dteday'' e ``mnth'', bem como
o valor do VIF associado às variáveis independentes restantes, podemos
concluir que:

\begin{enumerate}
\def\labelenumi{\arabic{enumi})}
\item
  \textbf{Trata-se de um modelo globalmente significativo} e ao menos
  uma variável independente é significativa para explicar a variável
  dependente ``cnt''. Tal assunção é corroborada pelo valor de
  F-statistic, o qual encontra-se distante de 1 (F-Statistic = 180.1),
  tendo ainda o p-value associado a este teste um valor \textless{}
  \(\alpha\) (p-value: \textless{} 2.2e-16)
\item
  O pressuposto da não existência de multicolinearidade entre as
  variáveis independentes não foi violado. \textbf{Portanto, não há
  multicolinearidade no modelo em questão.}
\item
  Os resíduos possuem distribuição muito próxima da normal e sua média
  aproxima-se sobremaneira de 0, conforme demonstrado no histograma a
  seguir:
\end{enumerate}

\begin{figure}[H]

{\centering \includegraphics{R_SOURCE_CODE_BIKE_SHARING_files/figure-pdf/unnamed-chunk-28-1.pdf}

}

\caption{Distribuições dos Resíduos}

\end{figure}%

\begin{enumerate}
\def\labelenumi{\arabic{enumi})}
\setcounter{enumi}{3}
\tightlist
\item
  O modelo possui uma boa acurácia quando aplicado aos dados de treino,
  haja vista o coeficiente de determinação ajustado = 0.8328.
\end{enumerate}

Resta-nos entretanto aplicá-lo aos dados de teste para saber qual será a
acurácia para dados não vistos anteriormente e saber se há um possível
overfitting, ou até mesmo underfitting.

\paragraph{Aplicação do último modelo aos dados de
teste:}\label{aplicauxe7uxe3o-do-uxfaltimo-modelo-aos-dados-de-teste}

\begin{verbatim}
[1] "R-squared: 0.776348088959387"
\end{verbatim}

\begin{verbatim}
[1] "R-squared (Adjusted): 0.747489777857372"
\end{verbatim}

Apesar de o modelo apresentar uma ligeira queda no valor do coeficiente
de determinação ajustado em relação ao primeiro modelo que fora definido
neste estudo, é possível ainda assim dizer que trata-se de um bom
coeficiente (R-squared (Adjusted) = 0.7474). Logo, pode-se dizer que o
modelo está apto a fazer boas predições para novos dados, e que este é
melhor por não possuir os problemas de multicolinearidade e excesso de
variáveis que os dois anteriores possuíam.

Analisemos a seguir o quão bem ajustada está a reta de regressão do
modelo em questão:

\begin{center}
\includegraphics{R_SOURCE_CODE_BIKE_SHARING_files/figure-pdf/unnamed-chunk-30-1.pdf}
\end{center}

O gráfico em questão confirma a assunção de que o modelo 3, o qual não
conta com as variáveis preditoras ``workingday'', ``dteday'' e ``mnth'',
é adequado para prever o número de bicicletas que serão alugadas,
utilizand-se para tanto do conjunto de variáveis independentes restantes
no conjunto de dados bike sharing, as quais são significativas para o
contexto do nosso modelo.

\subsection{Extra - Avaliação da performance de uma árvore de decisão
para a mesma tarefa de
Regressão.}\label{extra---avaliauxe7uxe3o-da-performance-de-uma-uxe1rvore-de-decisuxe3o-para-a-mesma-tarefa-de-regressuxe3o.}

Após desenvolver e analisar a performance de 3 modelos de Regressão
Linear Múltipla para a tarefa de previsão do número total de bicicletas
alugadas/dia em função de uma série de variáveis independentes,
testaremos como será a performance de uma árvore de decisão para o mesmo
conjunto de dados.

\begin{verbatim}
[1] 216  13
\end{verbatim}

Conforme observamos acima, 216 observações do nosso dataframe foram
separadas para Teste do modelo de árvore de decisão. As observaçòes
restantes constituem o conjunto de Treino do modelo.

\begin{figure}[H]

{\centering \includegraphics{R_SOURCE_CODE_BIKE_SHARING_files/figure-pdf/unnamed-chunk-32-1.pdf}

}

\caption{Árvore com 8 Nós-Folha}

\end{figure}%

\begin{verbatim}

Regression tree:
tree(formula = cnt ~ ., data = df_limpo, subset = traint.df)
Variables actually used in tree construction:
[1] "temp"   "season" "yr"     "atemp"  "hum"   
Number of terminal nodes:  8 
Residual mean deviance:  737800 = 363700000 / 493 
Distribution of residuals:
   Min. 1st Qu.  Median    Mean 3rd Qu.    Max. 
-5120.0  -438.2   117.3     0.0   524.5  2140.0 
\end{verbatim}

Ao treinarmos o modelo aplicando-o ao data set de treino, obtivemos como
output automático do R uma árvore de decisão com 8 nós-folha.

Isto está diretamente ligado à complexidade do modelo. Sendo assim,
quanto maior o número de nós folha, mais complexo se torna o modelo, o
que pode levá-lo ao sobreajuste.

É salutar ressaltar que o R utilizou como variáveis independentes apenas
o conjunto ``temp'', ``season'', ``yr'', ``atemp'' e ``hum''.

Vejamos a capacidade de realizar previsões do nosos modelo e a distância
entre o observado e aquilo que foi predito para os dados de teste:

\includegraphics{R_SOURCE_CODE_BIKE_SHARING_files/figure-pdf/unnamed-chunk-33-1.pdf}

O RMSE (Root Mean Square Error) é uma medida que quantifica a média das
diferenças entre os valores previstos por um modelo e os valores reais
de uma variável de interesse. É comumente usado para avaliar a precisão
de modelos de previsão, como modelos de regressão, \textbf{onde valores
menores indicam uma melhor capacidade de previsão do modelo.}

Com vistas a melhorar a capacidade preditiva do modelo, utilizaremos o
mecanismo da validação cruzada para tentar encontrar o número ideal de
nós-folha para o nosso modelo de árvore de decisão para regressão.

Para a primeira árvore com 8 nós-folha, alcançamos um RMSE de 1005.35.

\subsection{Aplicação da validação cruzada e avaliação dos
resultados:}\label{aplicauxe7uxe3o-da-validauxe7uxe3o-cruzada-e-avaliauxe7uxe3o-dos-resultados}

\begin{figure}[H]

{\centering \includegraphics{R_SOURCE_CODE_BIKE_SHARING_files/figure-pdf/unnamed-chunk-34-1.pdf}

}

\caption{Erro de Validação Cruzada vs.~Tamanho da Árvore}

\end{figure}%

Mesmo que a validação cruzada não aponte uma grande diferença de RMSE da
árvore com 6 ou 8 nós, é importante que façamos o treinamento do modelo
com 6 nós-folha com o objetivo de diminuir a complexidade do modelo e
saber qual será o tamanho da perda, em termos de precisão, uma vez que a
depender deste resultado pode ser mais vantajoso utilizar o modelo com
menos nós.

Um nó folha (nó terminal) é o ponto final onde uma decisão ou
classificação é feita. Ele não possui ramificações adicionais,
representando uma conclusão ou resultado final baseado nos atributos e
critérios analisados ao longo do caminho na árvore de decisão.

\textbf{Desta feita, após a avaliacão dos resultados, treinaremos uma
nova árvore de decisão, desta vez com 6 nós-folha e avaliaremos a
performance das suas previsões por meio da comparação do valor da raíz
quadrada do erro médio(RMSE):}

\begin{figure}[H]

{\centering \includegraphics{R_SOURCE_CODE_BIKE_SHARING_files/figure-pdf/unnamed-chunk-35-1.pdf}

}

\caption{Nova árvore com apenas 6 Nós-Folha}

\end{figure}%

\begin{verbatim}
[1] "RMSE da árvore com 6 Nós-Folha: 1033.20818180359"
\end{verbatim}

Como é possível perceber, a diminuição da complexidade da árvore por
meio da poda, levou o modelo a uma perda significativa de RMSE, tendo
este resultado em 1033.20, frente aos 1005.35 da árvore com 8 nós.

Para diminuir a complexidade da árvore, o R eliminou duas variáveis
independentes deste último modelo, sendo elas ``atemp'' e ``hum'', o que
pode ter levado ainda ao aumento do RMSE.

Haja vista que o desenvolvimento da árvore de decisão apresenta-se neste
estudo como um ``extra'', sendo o seu principal objeto as regressões
lineares múltiplas, utilizaríamos a primeira árvore de decisão se
quisessemos realizar previsões por meio deste tipo de modelo (decision
trees) e não a última, com apenas 6 nós-folha.

\section{Conclusão}\label{conclusuxe3o}

Concluímos o presente projeto com a assunção de que o último modelo de
regressão linear múltipla desenvolvido (modelo 3) foi o que mais se
aproximou daquilo que consideramos ideal para prever novos valores para
o número de alugueis de bicicletas, dadas as mesmas informações
constantes ao data set bike sharing.

Desenvolvemos e construímos o modelo tendo por base as etapas iniciais
de limpeza e exploração dos dados, passando posteriormente pela
verificação dos pressupostos básicos para o desenvolvimento do modelo de
regressào linear ,múltipla, divisão dos dados entre treino e teste e,
por fim, a comparação de modelos até chegar àquele que consideramos o
que melhor generaliza sobre os dados e os padrões relacionados, tendo
como benchmark para comparação de modelos o coeficiente de determinação
ajustado.

Em seguida, como uma etapa extra, desenvolvemos duas árvores de decisão
com vistas a efetuar a mesma tarefa de regressão para a variável
dependente CNT do nosos conjunto de dados, senod uma com 8 nós-folha (a
qual seria escolhida caso fossemos optar neste estudo por modleos de
árvore de decisão) e outra com apenas 6 nós-folha.

Desta feita, como próximas etapas do presente projeto poderíamos citar a
aplicação e desenvolvimento de novos modelos em árvore para o mesmo
conjunto de dados, como por exemplo Random Forest ou XGboost, com vistas
a compará-los a tudo o que fora desenvolvido neste estudo e buscar
aquele que melhor se adeque aos dados, sem contudo perder a capacidade
de generalizar e prever sobre novos dados de teste e/ou validação..

Por fim, após a escolha do melhor modelo baseado em critérios objetivos
seria possível construir a etapa de deploy para colocá-lo em produção em
benefício da organização, ao utilizar para tanto ferramentas cloud e
esteiras de CI/CD.

\textbf{O código fonte em R do presente trabalho consta ao link do
reposiório a seguir:
\href{https://github.com/mello-pedro/SUPERV_ML_MCDE_23_24/blob/main/R_SOURCE_CODE_BIKE_SHARING.qmd}{Repositório
GitHub}.}



\end{document}
